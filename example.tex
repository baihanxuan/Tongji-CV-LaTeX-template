\documentclass[11pt]{article}

\usepackage{hyperref}
\usepackage{xcolor}
\usepackage{calc}
\usepackage{graphicx}
\usepackage{tikz}
\usepackage[fontset=none]{ctex}
\usepackage{titlesec}
\usepackage{enumitem}
\usepackage{fancybox}

% 使用了 Google Fonts 的 Material Symbols,可以在
% https://fonts.google.com/icons 查找图标在文中使用,比如
% "Check Circle" 图标可以通过 \mSymbol{check-circle} 命令绘制
\usepackage{material-symbols}

\hypersetup{hidelinks}

%%%%%%%%%%%%%%%%%%%%
% 设置
%%%%%%%%%%%%%%%%%%%%

\setlength{\parindent}{0pt}					% 取消全局段落缩进
\pagenumbering{gobble}						% 取消页码显示
\setlist[itemize]{nosep                     % 取消 itemize 的默认间距
    , before={\vspace*{-\parskip}}          % 取消 itemize 和后续段落之间的空白
    , leftmargin=*}		                    % 取消 itemize 的左边距
\setlist[enumerate]{leftmargin=*}	        % 取消 enumerate 的左边距
\renewcommand{\arraystretch}{1.2}           % 设置表格行间距
\linespread{1.25}                           % 设置正文行间距

\titleformat{\section}					    % 将原标题前面的数字取消了
  {\LARGE\bfseries\raggedright} 		      % 字体改为 LARGE,bold,左对齐
  {}{0em}                      			  % 可用于添加全局标题前缀
  {}                           			  % 可用于添加代码
  [{\color{secondary_color}\titlerule}]     % 标题下方加一条线
\titlespacing*{\section}{0cm}{*1.2}{*1.2}	% 标题左边留白,上方,下方

\usepackage[
	a4paper,
	left=1.2cm,
	right=1.2cm,
	top=1.5cm,
	bottom=1cm,
	nohead
]{geometry}                                 % 页面边距设置

% 字体设置
\setmainfont[
    Path=fonts/,
    Extension=.otf,
    BoldFont=*-Bold,
]{NotoSerifSC}

\setCJKmainfont[
Path=fonts/,
Extension=.otf,
BoldFont=*-Bold,
]{NotoSerifSC}

% 自定义颜色(参考 https://github.com/seumxc/SEU-Logo)
% 完全没有找到同济大学 VI 设计手册的 PDF 文件,所以只能参考这里了:https://baike.baidu.com/item/同济大学校徽/16301175
\definecolor{primary_color}{RGB}{0, 102, 255}    % C100 M60 Y0 K40
\definecolor{secondary_color}{RGB}{0, 0, 0}

\newlength{\iconwidth}
\setlength{\iconwidth}{1.5em}                   % 设置 section 标题部分图标占用的宽度

%%%%%%%%%%%%%%%%%%%%
% 文章内容
%%%%%%%%%%%%%%%%%%%%

% 学院
\newcommand{\school}{下北泽学院 | Shimokitazawa School} 

% 联系方式
\newcommand{\contact}{
    % 根据个人喜好选择字号
    \footnotesize
    \textcolor{white}{
        % 邮箱
        \href{mailto:youremail@tongji.edu.cn}{{\normalsize\mSymbol{mail}}\quad youremail@tongji.edu.cn}
        \hspace{4em}
        % 手机号
        \href{tel:03 1145 1419}{{\normalsize\mSymbol{call}}\quad 03\ 1145\ 1419}
        % 别的联系方式,如微信、GitHub等
        \hspace{4em}
        \href{https://github.com/baihanxuan/Tongji-CV-LaTeX-template}{{\normalsize\mSymbol{package-2}}\quad GitHub 项目地址}
    }
}

\begin{document}

    %%%%%%%%%%%%%%%%%%%%
    % 页眉、页脚和背景(如果有多页简历,请把页眉页脚和背景复制粘贴到第二页的内容之前)
    %%%%%%%%%%%%%%%%%%%%

    % 页眉:校标组合+学院名
    \begin{tikzpicture}[remember picture, overlay]
        \node[anchor=north, inner sep=0pt](header) at (current page.north){
            \includegraphics[width=\paperwidth]{images/header.png}
        };
        \node[anchor=west](school_logo) at (header.west){
            \hspace{0.5cm}
            \includegraphics[width=0.2\textwidth]{images/logowhite.png}
        };
        \node[anchor=east](school_name) at(header.east){
            \textcolor{white}{\textbf{\school}}
            \hspace{0.5cm}
        };
    \end{tikzpicture}
    \vspace{-3.5em}

    % 页脚,联系方式
    \begin{tikzpicture}[remember picture, overlay]
        \node[anchor=south, inner sep=0pt](footer) at (current page.south){
            \includegraphics[width=\paperwidth]{images/footer.png}
        };
        % 联系方式
        \node[anchor=center] at(footer.center){\contact};
    \end{tikzpicture}

    % 背景
    \begin{tikzpicture}[remember picture, overlay]
        \node[opacity=0.05] at(current page.center){
            \includegraphics[width=0.7\paperwidth, keepaspectratio]{images/tongji-badge.png}
        };
    \end{tikzpicture}

    %%%%%%%%%%%%%%%%%%%%
    % 简历正文
    %%%%%%%%%%%%%%%%%%%%

    \begin{minipage}[t]{0.78\textwidth}
        % 个人信息
        \begin{minipage}[t]{\textwidth}
        \section[个人信息]{\makebox[\iconwidth][c]{\color{primary_color}{\Huge\mSymbol{badge}}}\quad 个人信息}
        \begin{minipage}[t]{0.5\textwidth}
            \textbf{姓\qquad 名}:李田所
            
            \vspace{0.5em}
            \textbf{出生年月}:2114年5月14日
        \end{minipage}
        \begin{minipage}[t]{0.35\textwidth}
            \textbf{性\qquad 别}:homo
            
            \vspace{0.5em}
            \textbf{地\qquad 区}:日本\ 东京都\ 下北泽
        \end{minipage}
        \vspace{1.2em}
        \end{minipage}

        % 教育背景
        \begin{minipage}[t]{\textwidth}
        \section[教育背景]{\makebox[\iconwidth][c]{\color{primary_color}{\Huge\mSymbol{school}}}\quad 教育背景}
        
        {\large \textbf{同济大学}},专科 \hfill 2024年9月--2028年6月
        \begin{itemize}
            \item 下北泽学院,一个一个专业
            \item \textbf{主修课程}:非常的新鲜、非常的美味\ 等。
        \end{itemize}
        
        \vspace{0.5em}
        {\large \textbf{同济大学}},本科 \hfill 2028年9月--2032年6月
        \begin{itemize}
            \item 野兽学院,十分甚至九分的专业
            \item \textbf{主修课程}:野兽数学、先辈数学、哼哼、啊啊啊啊啊啊\ 等。
            \item \textbf{GPA}:5.0 / 5.0(排名:10 / 9)
        \end{itemize}
        
        \vspace{0.5em}
        {\large \textbf{同济大学}},硕士 \hfill 2032年6月--至今
        \begin{itemize}
            \item 先辈学院,野兽专业,田所\ 浩二\ 教授
            \item \textbf{研究方向}:豪俊金曲\ 等。
        \end{itemize}
        
        \vspace{1.2em}
        \end{minipage}
    \end{minipage}
    \hfill
    % 右半边,照片,比例占行宽20%
    \begin{minipage}[t]{0.2\textwidth}
        \vspace{2em} % 照片上侧内容
        \setlength{\fboxsep}{0pt}
        \doublebox{\includegraphics[width=\linewidth]{images/avatar.png}}
    \end{minipage}

    \begin{minipage}[t]{\textwidth}
    % 科研成果
    \section[科研成果]{\makebox[\iconwidth][c]{\color{primary_color}{\Huge\mSymbol{contract-edit}}}\quad 科研成果}

    % 科研著作(研究生)
    9 > 10 is All You Need
    \begin{itemize}
        \item \textbf{Li Tiansuo}, Yajuu Senpai. \hfill 发表于 \textbf{Beast Senpai Conference}(CCF-A类会议)
        \item 推翻了 $9 < 10$ 的数学大厦,科普了 $9 > 10$ 的一个一个一个野兽定理
    \end{itemize}

    \vspace{0.5em}
    《昏睡红茶的制备机理》
    \begin{itemize}
        \item  \textbf{李田所}、田所、田所\ 浩二 \hfill 发表于 \textbf{会员制期刊} (SCI-1区)
        \item 提出了一种工业制备昏睡红茶的方法,测试产率达到\ 114.514\%
    \end{itemize}
    
    \vspace{1.2em}
    \end{minipage}

    \begin{minipage}[t]{\textwidth}
    % 项目经历\科研经历\项目与教学(标题请根据需要修改)
    \section[项目与教学]{\makebox[\iconwidth][c]{\color{primary_color}{\Huge\mSymbol{co-present}}}\quad 项目与教学}
    
    {\large \textbf{SITP 项目:基于一个东西的一个应用}} \hfill 2025年9月--2028年9月
    \begin{itemize}
        \item \textbf{第一主持人} \hfill 横向/纵向项目-已完结/进行中
        \item 每天搭乘黑色高级车去会员制餐厅用餐,获得了持续的欣快感
    \end{itemize}

    \vspace{0.5em}
    {\large \textbf{卓越星主题讨论班}},主讲 / 参与 \hfill 2026年夏季
    \begin{itemize}
        \item \textbf{主要内容}:同学报一下学号,给你加力行之星\ 等。
    \end{itemize}

    \vspace{0.5em}
    {\large \textbf{Zundoko Veron Cho}},助教 \hfill 2032年夏季
    \begin{itemize}
        \item \textbf{主要内容}:世界怪奇物语真好看\ 等。
    \end{itemize}
    
    \vspace{1.2em}
    \end{minipage}
    
    % 如果每行的内容不是很多,可以考虑使用 minipage,将内容分列展示
    \begin{minipage}[t]{0.6\textwidth}
        \section[技能特长]{\makebox[\iconwidth][c]{\color{primary_color}{\Huge\mSymbol{handyman}}}\quad 技能特长}
        \begin{itemize}
        \setlength{\itemsep}{0.5em}
            \item 熟练使用 Python、Jvav、Rust 等编程语言。
            \item 熟练使用 Tensorflow、Pytorch 等深度学习框架。
            \item 熟悉 Windows 与 Linux 端开发。
        \end{itemize}
    \end{minipage}
    \hfill
    \begin{minipage}[t]{0.35\textwidth}
        \section[兴趣爱好]{\makebox[\iconwidth][c]{\color{primary_color}{\Huge\mSymbol{star}}}\quad 兴趣爱好}
        \begin{itemize}
        \setlength{\itemsep}{0.5em}
            \item 我喜欢唱
            \item 跳
            \item Rap
            \item 篮球
        \end{itemize}
    \end{minipage}
    
    % \newpage
    % % 如有需要,可以添加额外的页面。不要忘记添加页眉页脚和背景相关的代码。
    
\end{document}
